%===================================== CHAP 5 =================================

\chapter{Discussion}
The results presented are weakened by the fact that no position ground truth, using e.g. GNSS, was generated throughout the experiments, and the results presented in the previous chapter assumes that the point cloud data generated represent the true location of the boat. When examining the data, however, there is reason to believe that this is not inaccurate. The tracks generated by the point cloud cluster centroids are a realistic representation of the maneuvers performed, and the tracks are coherent with no observable large jumps or discontinuities. No returns not believed to originate from the target boat were found in the lidar data. It should be noted that the water surface was calm at the time of the experiments, rough seas could potentially produce a different scenario.
\subsection{Detection of Boats in Images}
As the results show, there are some places where the detection of boats in the images suffer. Upon examining the images where detection largely failed, for many of them this is the case when boats in the background are close to the target boat in the image, leading them to be detected as one. Figure \ref{fig:issues_multiple_detections} shows two cases of this. 
\begin{figure}[H]
	\centering
	\begin{subfigure}[t]{.5\textwidth}
		\centering
		\includegraphics[width=\linewidth]{fig/issue_overlapping.png}
		\caption{Three ships detected as one (far left of the\\ image).}
		\label{fig:sub_issue1}
	\end{subfigure}%
	\begin{subfigure}[t]{.5\textwidth}
		\centering
		\includegraphics[width=\linewidth]{fig/issue_2.png}
		\caption{Two ships detected as one (second bounding box from the left).}
		\label{fig:sub_issue2}
	\end{subfigure}
	\caption{Two examples of multiple ships being detected and classified as one.}
	\label{fig:issues_multiple_detections}
\end{figure}
This could have been avoided by using e.g. background subtraction, removing the static background in the image \cite{modernCV}, so that no boats other than the target boat was visible in the images. It is, however, an interesting observation and a potential weakness in the proposed visual detection model in a realistic scenario.

In evaluating the classification results for the visual detection, it would be useful to have hand-generated a ground truth bounding box for the target boat. The large size of the datasets made this prohibitively time consuming, as a total of 12094 images were generated. The goal of the project was to compare detections between the sensors, and spot checks taken 

\begin{figure}
	\centering
	\begin{subfigure}[t]{.5\textwidth}
	\centering
	\includegraphics[width=\linewidth]{fig/discussion/ex1_1.png}
	\end{subfigure}%
	\begin{subfigure}[t]{.5\textwidth}
	\centering
	\includegraphics[width=\linewidth]{fig/discussion/ex2_1.png}
	\end{subfigure}

	\begin{subfigure}[t]{.5\textwidth}
	\centering
	\includegraphics[width=\linewidth]{fig/discussion/ex3_1.png}
	\end{subfigure}%
	\begin{subfigure}[t]{.5\textwidth}
	\centering
	\includegraphics[width=\linewidth]{fig/discussion/ex4_1.png}
	\end{subfigure}

	\begin{subfigure}[t]{.5\textwidth}
	\centering
	\includegraphics[width=\linewidth]{fig/discussion/ex5_1.png}
	\end{subfigure}%
	\begin{subfigure}[t]{.5\textwidth}
	\centering
	\includegraphics[width=\linewidth]{fig/discussion/ex6_1.png}
	\end{subfigure}
	\caption{Examples of successfull detections of the target boat.}
	\label{some example}
\end{figure}
\cleardoublepage