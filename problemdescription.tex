\clearpage
\pagenumbering{roman} 				
\setcounter{page}{1}

\pagestyle{fancy}
\fancyhf{}
\renewcommand{\chaptermark}[1]{\markboth{\chaptername\ \thechapter.\ #1}{}}
\renewcommand{\sectionmark}[1]{\markright{\thesection\ #1}}
\renewcommand{\headrulewidth}{0.1ex}
\renewcommand{\footrulewidth}{0.1ex}
\fancyfoot[LE,RO]{\thepage}
\fancypagestyle{plain}{\fancyhf{}\fancyfoot[LE,RO]{\thepage}\renewcommand{\headrulewidth}{0ex}}

\section*{\LARGE Problem Description}	
$\\[0.5cm]$

\noindent
The goal of the specialization project is to prepare the processing pipeline leading to camera-lidar fusion. A comparison of detections of interesting objects in the camera data with interesting objects in the lidar data will be performed, and suitable detection models for camera and lidar will be analyzed \todo{Hvis tid}.\smallskip \\
The following subtasks are proposed for the project:
\begin{enumerate}
	\item Installation of Velodyne lidar together with camera, and recording of time-synchronized data from both sensors.
	\item Calibration of the sensors, including specification of world-frame-to-sensor-frame measurement models for both sensors.
	\item Implementation of a detector based on a convolutional neural network (CNN) for the detection of boats in camera data.
	\item Implementation of a detector based on intensity of reflected signal strength for the lidar.
	\item Analysis of the extent to which lidar detections correspond to camera detections and vice versa.
\end{enumerate}


\clearpage